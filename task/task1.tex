\documentclass[a4paper,12pt]{article}

% Packages
\usepackage[utf8]{inputenc} 
\usepackage[T1]{fontenc} 
\usepackage{graphicx}
\usepackage{setspace}
\usepackage{titlesec}
\usepackage{geometry}
\usepackage{ragged2e} % For justified text
\usepackage{svg}
\geometry{a4paper, margin=2.5cm}

% Formatting
\setstretch{1.5}
\titleformat{\section}[block]{\bfseries\Large}{}{0em}{}
\titleformat{\subsection}[block]{\bfseries\large}{}{0em}{}
\titleformat{\subsubsection}[block]{\bfseries\normalsize}{}{0em}{}

\begin{document}

% Title Section
\begin{flushleft}
{\bfseries\Large Werkzeuge für das wissenschaftliche Arbeiten}\\
\normalsize Python for Machine Learning and Data Science\\
\normalfont\hspace*{4.5cm}Abgabe: 15.12.2023\\
\hrulefill
\end{flushleft}

% Table of Contents
\section*{Inhaltsverzeichnis}
\addcontentsline{toc}{section}{Inhaltsverzeichnis}
Inhaltsverzeichnis \dotfill Seite 1\\
\textbf{1. Projektaufgabe} \dotfill Seite 1\\
1.1. Einleitung \dotfill Seite 1\\
1.2. Aufbau \dotfill Seite 2\\
1.3. Methoden \dotfill Seite 2\\
\textbf{2. Abgabe} \dotfill Seite 3\\
\hrulefill

% Main Content
\section{1. Projektaufgabe}
\justify
In dieser Aufgabe beschäftigen wir uns mit Objektorientierung in Python. Der Fokus liegt auf der Implementierung einer Klasse, dabei nutzen wir insbesondere auch Magic Methods.

\begin{figure}[h!]
    \centering
    \includesvg[width=12cm, height=5cm]{diagram/classes_files.svg}
    \caption*{\scriptsize\bfseries Abbildung 1: \normalfont Darstellung der Klassenbeziehungen.}
\end{figure}

\subsection{1.1. Einleitung}
\justify
Ein Datensatz besteht aus mehreren Daten, ein einzelnes Datum wird durch ein Objekt der Klasse "DataSetItem" repräsentiert. Jedes Datum hat einen Namen (Zeichenkette), eine ID (Zahl) und beliebigen Inhalt.

Nun sollen mehrere Daten, Objekte vom Typ "DataSetItem", in einem Datensatz zusammengefasst werden. Sie haben sich schon auf eine Schnittstelle und die benötigten Operationen, die ein Datensatz unterstützen muss, geeinigt. Es gibt eine Klasse "DataSetInterface", die die Schnittstelle definiert und Operationen jedes Datensatzes angibt. Bisher fehlt aber noch die Implementierung eines Datensatzes mit allen Operationen.

Implementieren Sie eine Klasse "DataSet" als eine Unterklasse von "DataSetInterface".

\subsection{1.2. Aufbau}
\justify
Es gibt drei Dateien: "dataset.py", "main.py" und "implementation.py".
\begin{itemize}
    \item In der "dataset.py" befinden sich die Klassen "DataSetInterface" und "DataSetItem".
    \item In der Datei "implementation.py" muss die Klasse "DataSet" implementiert werden.
    \item Die Datei "main.py" nutzt die Klassen "DataSet" und "DataSetItem" aus den jeweiligen Dateien und testet die Schnittstelle und Operationen von "DataSetInterface".
\end{itemize}

\subsection{1.3. Methoden}
\justify
Bei der Klasse "DataSet" sind insbesondere folgende Methoden zu implementieren. Die genaue Spezifikation finden Sie in der "dataset.py":
\begin{itemize}
    \item \_\_setitem\_\_(self, name, id\_content): Hinzufügen eines Datums mit Name, ID und Inhalt.
    \item \_\_iadd\_\_(self, item): Hinzufügen eines "DataSetItem".
    \item \_\_delitem\_\_(self, name): Löschen eines Datums auf Basis des Namens. Jeder Name ist ein eindeutiger Schlüssel und darf nur einmal pro Datensatz vorkommen.
    \item \_\_contains\_\_(self, name): Prüfung, ob ein Datum mit diesem Namen im Datensatz vorhanden ist.
    \item \_\_getitem\_\_(self, name): Abrufen des Datums über seinen Namen.
    \item \_\_and\_\_(self, dataset): Schnittmenge zweier Datensätze bestimmen und als neuen Datensatz zurückgeben.
    \item \_\_or\_\_(self, dataset): Vereinigungen zweier Datensätze bestimmen und als neuen Datensatz zurückgeben.
    \item \_\_iter\_\_(self): Iteration über alle Daten des Datensatzes (optional mit Sortierung für die Reihenfolge).
    \item filtered\_iterate(self, filter): Gefilterte Iteration über einen Datensatz mittels Lambda-Funktion mit Parametern Name und ID.
    \item \_\_len\_\_(self): Anzahl der Daten im Datensatz abrufen.
\end{itemize}

\section{2. Abgabe}
\justify
Programmieren Sie die Klasse "DataSet" in der Datei "implementation.py", um die oben beschriebene Aufgabe im VPL zu lösen. Sie können auch direkt auf Ihrem Computer programmieren; dazu finden Sie alle drei benötigten Dateien zum Download im Moodle.

Das VPL nutzt den gleichen Code, wobei die "main.py" um weitere Testfälle und Überprüfungen erweitert wurde. Die Überprüfungen dienen dazu sicherzustellen, dass Sie die richtigen Klassen nutzen.

\vspace{1cm}
\hrulefill
\vspace{0.5cm}
\footnotesize $^*$ Dateien befinden sich im Ordner "code/" dieses Git-Repositories.

\end{document}
