\documentclass[a4paper,12pt]{article}

% Packages
\usepackage[utf8]{inputenc} 
\usepackage[T1]{fontenc} 
\usepackage{graphicx}
\usepackage{setspace}
\usepackage{titlesec}
\usepackage{geometry}
\usepackage{ragged2e} % For justified text
\geometry{a4paper, margin=2.5cm}

% Formatting
\setstretch{1.5}
\titleformat{\section}[block]{\bfseries\Large}{}{0em}{}
\titleformat{\subsection}[block]{\bfseries\large}{}{0em}{}
\titleformat{\subsubsection}[block]{\bfseries\normalsize}{}{0em}{}

\begin{document}

% Title Section
\begin{flushleft}
{\bfseries\Large Werkzeuge f\"ur das wissenschaftliche Arbeiten}\\
\normalsize Python for Machine Learning and Data Science\\
\normalfont\hspace*{4.5cm}Abgabe: 15.12.2023\\
\hrulefill
\end{flushleft}

% Table of Contents
\section*{Inhaltsverzeichnis}
\addcontentsline{toc}{section}{Inhaltsverzeichnis}
Inhaltsverzeichnis \dotfill Seite 1\\
\textbf{1. Projektaufgabe} \dotfill Seite 1\\
1.1. Einleitung \dotfill Seite 1\\
1.2. Aufbau \dotfill Seite 2\\
1.3. Methoden \dotfill Seite 2\\
\textbf{2. Abgabe} \dotfill Seite 3\\
\hrulefill

% Main Content
\section{1. Projektaufgabe}
\justify
In dieser Aufgabe besch\"aftigen wir uns mit Objektorientierung in Python. Der Fokus liegt auf der Implementierung einer Klasse, dabei nutzen wir insbesondere auch Magic Methods.

\begin{figure}[h!]
    \centering
    \includegraphics[width=12cm, height=5cm]{./../diagram/classes_files.png}
    \caption*{\scriptsize\bfseries Abbildung 1: \normalfont Darstellung der Klassenbeziehungen.}
\end{figure}

\subsection{1.1. Einleitung}
\justify
Ein Datensatz besteht aus mehreren Daten, ein einzelnes Datum wird durch ein Objekt der Klasse "DataSetItem" repr\"asentiert. Jedes Datum hat einen Namen (Zeichenkette), eine ID (Zahl) und beliebigen Inhalt.

Nun sollen mehrere Daten, Objekte vom Typ "DataSetItem", in einem Datensatz zusammengefasst werden. Sie haben sich schon auf eine Schnittstelle und die ben\"otigten Operationen, die ein Datensatz unterst\"utzen muss, geeinigt. Es gibt eine Klasse "DataSetInterface", die die Schnittstelle definiert und Operationen jedes Datensatzes angibt. Bisher fehlt aber noch die Implementierung eines Datensatzes mit allen Operationen.

Implementieren Sie eine Klasse "DataSet" als eine Unterklasse von "DataSetInterface".

\subsection{1.2. Aufbau}
\justify
Es gibt drei Dateien: "dataset.py", "main.py" und "implementation.py".
\begin{itemize}
    \item In der "dataset.py" befinden sich die Klassen "DataSetInterface" und "DataSetItem".
    \item In der Datei "implementation.py" muss die Klasse "DataSet" implementiert werden.
    \item Die Datei "main.py" nutzt die Klassen "DataSet" und "DataSetItem" aus den jeweiligen Dateien und testet die Schnittstelle und Operationen von "DataSetInterface".
\end{itemize}

\subsection{1.3. Methoden}
\justify
Bei der Klasse "DataSet" sind insbesondere folgende Methoden zu implementieren. Die genaue Spezifikation finden Sie in der "dataset.py":
\begin{itemize}
    \item \_\_setitem\_\_(self, name, id\_content): Hinzuf\"ugen eines Datums mit Name, ID und Inhalt.
    \item \_\_iadd\_\_(self, item): Hinzuf\"ugen eines "DataSetItem".
    \item \_\_delitem\_\_(self, name): L\"oschen eines Datums auf Basis des Namens. Jeder Name ist ein eindeutiger Schl\"ussel und darf nur einmal pro Datensatz vorkommen.
    \item \_\_contains\_\_(self, name): Pr\"ufung, ob ein Datum mit diesem Namen im Datensatz vorhanden ist.
    \item \_\_getitem\_\_(self, name): Abrufen des Datums \"uber seinen Namen.
    \item \_\_and\_\_(self, dataset): Schnittmenge zweier Datens\"atze bestimmen und als neuen Datensatz zur\"uckgeben.
    \item \_\_or\_\_(self, dataset): Vereinigungen zweier Datens\"atze bestimmen und als neuen Datensatz zur\"uckgeben.
    \item \_\_iter\_\_(self): Iteration \"uber alle Daten des Datensatzes (optional mit Sortierung f\"ur die Reihenfolge).
    \item filtered\_iterate(self, filter): Gefilterte Iteration \"uber einen Datensatz mittels Lambda-Funktion mit Parametern Name und ID.
    \item \_\_len\_\_(self): Anzahl der Daten im Datensatz abrufen.
\end{itemize}

\section{2. Abgabe}
\justify
Programmieren Sie die Klasse "DataSet" in der Datei "implementation.py", um die oben beschriebene Aufgabe im VPL zu l\"osen. Sie k\"onnen auch direkt auf Ihrem Computer programmieren; dazu finden Sie alle drei ben\"otigten Dateien zum Download im Moodle.

Das VPL nutzt den gleichen Code, wobei die "main.py" um weitere Testf\"alle und \"Uberpr\"ufungen erweitert wurde. Die \"Uberpr\"ufungen dienen dazu sicherzustellen, dass Sie die richtigen Klassen nutzen.

\vspace{1cm}
\hrulefill
\vspace{0.5cm}
\footnotesize $^*$ Dateien befinden sich im Ordner "code/" dieses Git-Repositories.

\end{document}
